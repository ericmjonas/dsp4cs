\documentclass[10pt]{article}

\usepackage{amsmath}
\usepackage{mathrsfs,amsmath} 
\usepackage{amsfonts}
\usepackage{graphicx}
\usepackage{grffile}
\usepackage{fullpage}
\usepackage[draft]{fixme}
\usepackage{mathtools}

%\newcommand{\ftarrow}{\xlongleftrightarrow{\mathscr{F}}}
\newcommand{\ftarrow}{\stackrel{\mathscr{F}}{\longleftrightarrow}}

\title{DSP 4 CS} 
\author{Eric Jonas (jonas@eecs.berkeley.edu)}


\title{The Fourier Transform}
\begin{document}
\maketitle

Not just another basis

Use example from encyclopedia of mathematics


* Spectral analysis. What is the relationship between the Fourier
  Transform, the Discrete-Time Fourier Transform, and the FFT? "Why
  does matlab’s FFT() always give me complex outputs?" forward,
  inverse transforms.


\section{The Continuous-time Fourier Transform}


\section{The Discrete-time Fourier Transform}


\section{The Discrete Fourier Transform}


\section{The Fast Fourier Transform}


\section{Analytic Signals}
Most signals in the real world are real-valued. 

Physicsists always say ``Well we can just use a complex signal and
then take the real part'' but

1. Why? 

2. Ok, there are a whole bunch of ways to create a complex signal from
a real one, why do we do it a certain way? 

3. I and Q -- quadrature signals

4. Negative Frequency




\end{document}
