\documentclass[10pt]{article}

\usepackage{amsmath}
\usepackage{mathrsfs,amsmath} 
\usepackage{amsfonts}
\usepackage{graphicx}
\usepackage{grffile}
\usepackage{fullpage}
\usepackage[draft]{fixme}
\usepackage{mathtools}

%\newcommand{\ftarrow}{\xlongleftrightarrow{\mathscr{F}}}
\newcommand{\ftarrow}{\stackrel{\mathscr{F}}{\longleftrightarrow}}

\title{DSP 4 CS} 
\author{Eric Jonas (jonas@eecs.berkeley.edu)}


\title{Signals and LTI Systems}
\begin{document}
\maketitle



\section{What is a signal }

A signal is a function of one or more independent variables, such as
$x(t)$, where we generally let $t$ range over $(-\infty, \infty)$. We
often think of such signals as representing some observation of the
world -- the voltage on a wire, the air pressure at a microphone, or
electromagnetic vibrations in the vacuum.

We will work with signals in continuous time, and represent them by
$x(t)$, $t \in \mathbb{R}$. We will also work with signals in discrete
time, $x[n]$, $n \in \mathbb{Z}$. Note that the behavior of these two
different classes of signals can be both subtle and profound, and
transitioning between one and the other implactes some of the most
classic results in signal processing, such as the Shannon/Nyquist
sampling theorem. However, a large number of the properties that apply
to discrete signals apply to continuous-time signals and vice versa,
so we will often state properties of one that apply equally to the
other. 


A great deal of ink has been spilled about which signals are
in some sense admissable, but we will gloss over most of those
issues here. We define the energy in a signal to be

\[
E_\infty = \int_{-\infty}^{\infty} |x(t)|^2 dt
\]

and note that many signals we work with will have infinite energy. 
There are mathematical formalisms you can use to handle all of this, 
but they are beyond the scope of this survey course. 

\subsection{Periodic Signals}
We will often deal with periodic signals, where
\[
x(t) = x(t+T)
\]

The canonical examples being sines and cosines. We will become intimately 
familiar with the periodic complex exponential signal, 
\[
x(t) = e^{j \omega t}
\]
which we remember from Kindergarden can be expressed via Euler's equation 
as 
\[
e^{j \omega_0 t} = \cos \omega_0 t + j \sin \omega_0 t
\]

Note that things become slightly more complex in the discrete case. We can
define
\fxwarning{Add discrete time example}

\subsection{the unit impulse}
The unit impulse, shows up in many areas of electrical
engineering, physics, and probability. We will first consider the 
discrete time version

The discrete-time unit impulse can be used to ``sample'' a signal
at time zero, as
\[
x[n]\delta[n] = x[0]\delta[n]
\]


\section{Analytic Signals}
Most signals in the real world are real-valued. 

Physicsists always say ``Well we can just use a complex signal and
then take the real part'' but

1. Why? 

2. Ok, there are a whole bunch of ways to create a complex signal from
a real one, why do we do it a certain way? 

3. I and Q -- quadrature signals

4. Negative Frequency

\section{Systems}

We speak of systems, which take in a signal $x(t)$ and produce a signal $y(t)$. WE
draw them like this. 

Electrical engineers and computer scientists can bond over our desire
to draw boxes and connect them with arrows. It's important to note that
a system can look at as much of the signal as it wants, but it also
must produce an output for every time. 

We could just add stuff together

We could have a little circuit and describe it with an ODE

\begin{itemize}
\item Memoryless 
\item invertable
\item causal
\item stability 
\item linearity
\item time-invariance 
\end{itemize}

\section{Convolution}
What is convolution. 

\[
y(t) = \int_{-\infty}^{\infty} x(\tau) h(t - \tau) d\tau
\]

``Flip and slide'' 

Asymptotic complexity 

\subsection{The linear algebra view}
Convolution as matrix multiplication (don't actually do this)


\section{Dirac Delta Function}

Basis of an LTI system

\end{document}
